\part*{Executive Summary}
\addcontentsline{toc}{part}{Executive Summary}

This document describes an accelerator benchmark suite, a set of 11 codes that includes 1 synthetic benchmarks and 10 comonly used applications. The key focus of this task has been exploiting accelerators or co-processors to improve the performance of real applications. It aims at providing a set of scalable, currently relevant and publically available codes and datasets.

This work has been undertaken be Task7.2B "Accelerator Benchmarks" in the PRACE Forth Implementation Phase (PRACE-4IP) project.

%selection
The selection process has been simple since most of the selected application are a subset of the Unified European Applications Benchmark Suite (UEABS) \ref{}. One application and synthetic benchmark suite have been added to complete it.

As a result, selected codes are: ALYA, Code_Saturne, CP2K, GROMACS,
GPAW, NAMD, PFARM, QCD, Quantum Espresso, SHOC and SPECFEM3D.

For each code either one or two test case datasets have been selected. These are described in
this document, along with a brief introduction to the application codes themselves. For each
code some sample results are presented, from from first run on leading egde systems.

% TODO
%Summarize the objectives and key results of the deliverable.
%PRACE deliverables document the work performed according to the Description of Action (Annex I to the Grant Agreement). All deliverables follow a consistent structure and layout. This prototype document describes in detail the conventions that the authors should adhere to. It is written as if it were a real deliverable. Delete the text in this deliverable bur retain the formatting.
%The description of the work should be as concise as possible. It should also be explicit and state the results clearly. The external reviewers do not appreciate if they have to read ‘between the lines’. The combination of clear message, well written text – without spelling and grammatical errors – and a pleasing layout are as important for the acceptance of the project achievements by the reviewers and the EC as the actual work.
%The Executive Summary should be less than one page. It should start on ‘Page 1’ and on the top page of the sheet.

%%%%%%%%%%%%%%%%%%%%%%%%%%%%%%%%%%%%
% Description of action/Grant Agreement
%%%%%%%%%%%%%%%%%%%%%%%%%%%%%%%%%%%%
% Task 7.2 will look further ahead to future PRACE exascale systems. A key focus of this task will be exploiting accelerators or co-processors to improve the performance of real applications. There are currently a variety of different classes of accelerators and we will compare these to demonstrate their relative strengths and weaknesses. As part of this activity, we will develop an accelerator benchmark suite that includes both synthetic benchmarks and a subset of the applications from the Unified European Applications Benchmark Suite (UEABS); this will help researchers interested in knowing how to use accelerators effectively for their application

\part{Introduction}
% Provide a brief description of the following (if appropriate):
% Objectives of the work related to the project as a whole;
% Purpose of the document;
% Structure of the document (what is in the different sections/chapters);
% Intended audience.


\part{Benchmark suite description}
%Description des codes/cas de tests/intérêt pour la communauté scientifique
% TODO: ou bien "as stated in WP212 brief code desc and test casess listing"
This part will cover each code, presenting the interest for the scientific comunity as well as the test cases defined for the benchmarks.
As an extention to the EUABS, \ref{} most of codes presented in this suite are included in the later. Exceptions are PFARM which come from PRACE-2IP \ref{} and SHOC a synthetic benchmark suite.
\section{Alya}
Alya is a high performance computational mechanics code that can solve different coupled mechanics problems: incompressible/compressible flows, solid mechanics, chemistry, excitable media, heat transfer and Lagrangian particle transport. It is one single code. There are no particular parallel or individual platform versions. Modules, services and kernels can be compiled individually and used a la carte. The main discretisation technique employed in Alya is based on the variational multiscale finite element method to assemble the governing equations into Algebraic systems. These systems can be solved using solvers like GMRES, Deflated Conjugate Gradient, pipelined CG together with preconditioners like SSOR, Restricted Additive Schwarz, etc. The coupling between physics solved in different computational domains (like fluid-structure interactions) is carried out in a multi-code way, using different instances of the same executable. Asynchronous coupling can be achieved in the same way in order to transport Lagrangian particles.

\subsection{Code desctiption}
The code is parallelised with MPI and OpenMP. Two OpenMP strategies are available, without and with a colouring strategy to avoid ATOMICs during the assembly step. A CUDA version is also available for the different solvers. Alya has been also compiled for MIC (Intel Xeon Phi).

Alya is written in Fortran 1995 and the incompressible fluid module, present in the benchmark suite, is freely available. This module solves the Navier-Stokes equations using an Orthomin \ref{} method for the pressure Schur complement. This method is an algebraic split strategy which converges to the monolithic solution. At each linearisation step, the momentum is solved twice and the continuity equation is solved once or twice according to if the momentum preserving or the continuity preserving algorithm is selected.

\subsection{Test cases desctiption}
\subsubsection{Cavity-hexaedra elements (10M elements)}
This test is the classical lid-driven cavity. The problem geometry is a cube of dimensions 1x1x1. The fluid properties are density=1.0 and viscosity=0.01. Dirichlet boundary conditions are applied on all sides, with three no-slip walls and one moving wall with velocity equal to 1.0, which corresponds to a Reynolds number of 100. The Reynolds number is low so the regime is laminar and turbulence modelling is not necessary. The domain is discretised into 9800344 hexaedra elements. The solvers are the GMRES method for the momentum equations and the Deflated Conjugate Gradient to solve the continuity equation. This test case can be run using pure MPI parallelisation or the hybrid MPI/OpenMP strategy.

\subsubsection{Cavity-hexaedra elements (30M elements)}
This is the same cavity test as before but with 30M of elements. Note that a mesh multiplication strategy enables one to multiply the number of elements by powers of 8, by simply activating the corresponding option in the ker.dat file.

\subsubsection{Cavity-hexaedra elements-GPU version (10M elements)}
This is the same test as Test case 1, but using the pure MPI parallelisation strategy with acceleration of the algebraic solvers using GPUs.

\section{Code Saturne}
Code Saturne is an open-source CFD software package developed by EDF R\&D since 1997 and open-source since 2007. The Navier-Stokes equations are discretised following a finite volume method approach. The code can handle any type of mesh built with any type of cell/grid structure. Incompressible and compressible flows can be simulated, with or without heat transfer, and a range of turbulence models is available. The code can also be coupled with itself or other software to model some multiphysics problems (fluid-structure, fluid-conjugate heat transfer, for instance).

\subsection{Code desctiption}
Parallelism is handled by distributing the domain over the processors (several partitioning tools are available, either internally, i.e. SFC Hilbert and Morton, or through external libraryies, i.e. METIS Serial, ParMETIS, Scotch Serial, PT-SCOTCH. Communications between subdomains are performed through MPI. Hybrid parallelism using OpenMP has recently been optimised for improved multicore performance.

For incompressible simulations, most of the time is spent during the computation of the pressure through Poisson equations. PETSc and HYPRE have recently been linked to the code to offer alternatives to the internal solvers to compute the pressure. The developer’s version of PETSc supports CUDA and will be used in this benchmark suite.

Code Saturne is written in C, F95 and Python. It is freely available under the GPL license.

\subsection{Test cases desctiption}
Two test cases are dealt with, the former with a mesh made of tetrahedral cells and the latter with a mesh made of hexahedral cells. Both configurations are meant for incompressible laminar flows. Note that both configurations will also be used in the regular UEABS

\subsubsection{Flow in a 3-D lid-driven cavity (tetrahedral cells)}
The geometry is very simple, i.e. a cube, but the mesh is built using tetrahedral cells. The Reynolds number is set to 400, and symmetry boundary conditions are applied in the spanwise direction. The case is modular and the mesh size can easily been varied. The largest mesh has about 13 million cells.

This test case is expected to scale efficiently to 1000+ nodes.

\subsubsection{3-D Taylor-Green vortex flow (hexahedral cells)}
The Taylor-Green vortex flow is traditionally used to assess the accuracy of CFD code numerical schemes. Periodicity is used in the 3 directions. The total kinetic energy (integral of the velocity) and enstrophy (integral of the vorticity) evolutions as a function of the time are looked at. Code Saturne is set for 2nd order time and spatial schemes, and three meshes are considered, containing 1283, 2563 and 5123 cells, respectively.

This test case is expected to scale efficiently to 4000+ nodes for the largest mesh.

\section{CP2K}
CP2K is a quantum chemistry and solid state physics software package that can perform atomistic simulations of solid state, liquid, molecular, periodic, material, crystal, and biological systems. It can perform molecular dynamics, metadynamics, Quantum Monte Carlo, Ehrenfest dynamics, vibrational analysis, core level spectroscopy, energy minimisation, and transition state optimisation using NEB or dimer method.

CP2K provides a general framework for different modeling methods such as DFT using the mixed Gaussian and plane waves approaches GPW and GAPW. Supported theory levels include DFTB, LDA, GGA, MP2, RPA, semi-empirical methods (AM1, PM3, PM6, RM1, MNDO, …), and classical force fields (AMBER, CHARMM, …).

\subsection{Code desctiption}
Parallelisation is achieved using a combination of OpenMP-based multi-threading and MPI.

Offloading for accelerators is implemented through CUDA and OpenCL for GPGPUs and through OpenMP for MIC (Intel Xeon Phi).

CP2K is written in Fortran 2003 and freely available under the GPL license.

\subsection{Test cases desctiption}
\subsubsection{LiH-HFX}
This is a single-point energy calculation for a particular configuration of a 216 atom Lithium Hydride crystal with 432 electrons in a 12.3 \ref{caractere pas bon ici}3 (Angstroms cubed) cell. The calculation is performed using a density functional theory (DFT) algorithm with Gaussian and Augmented Plane Waves (GAPW) under the hybrid Hartree-Fock exchange (HFX) approximation. These types of calculations are generally around one hundred times the computational cost of a standard local DFT calculation, although the cost of the latter can be reduced by using the Auxiliary Density Matrix Method (ADMM). Using OpenMP is of particular benefit here as the HFX implementation requires a large amount of memory to store partial integrals. By using several threads, fewer MPI processes share the available memory on the node and thus enough memory is available to avoid recomputing any integrals on-the-fly, improving performance.

This test case is expected to scale efficiently to 1000+ nodes.

\subsubsection{H2O-DFT-LS}
This is a single-point energy calculation for 2048 water molecules in a 39 \ref{caractere pas bon ici}3 box using linear-scaling DFT. A local-density approximation (LDA) functional is used to compute the Exchange-Correlation energy in combination with a DZVP MOLOPT basis set and a 300 Ry cutoff. For large systems the linear-scaling approach for solving Self-Consistent-Field equations should be much cheaper computationally than using standard DFT, and allow scaling up to 1 million atoms for simple systems. The linear scaling cost results from the fact that the algorithm is based on an iteration on the density matrix. The cubically-scaling orthogonalisation step of standard DFT is avoided and the key operation is sparse matrix-matrix multiplications, which have a number of non-zero entries that scale linearly with system size. These are implemented efficiently in CP2K's DBCSR library.

This test case is expected to scale efficiently to 4000+ nodes.

\section{GPAW}
GPAW is a software package for ab initio electronic structure calculations using the projector augmented wave (PAW) method. Using a uniform real-space grid representation of the electronic wavefunctions, as implemented in GPAW, allows for excellent computational scalability and systematic converge properties in density functional theory calculations.

\subsection{Code desctiption}
GPAW is written mostly in Python, but includes also computational kernels written in C as well as leveraging external libraries such as NumPy, BLAS and ScaLAPACK. Parallelisation is based on message-passing using MPI with no support for multithreading. Development branches for GPGPUs and MICs include support for offloading to accelerators using either CUDA or pyMIC/libxsteam, respectively. GPAW is freely available under the GPL license.

\subsection{Test cases desctiption}
\subsubsection{Carbon Nanotube}
This is a single-point energy calculation for a carbon nanotube (6,6) with a freely adjustable length (240 atoms by default). The calculation is performed using the residual minimisation method with the RMM-DIIS eigensolver and a multigrid Jacobian method as a Poisson solver.

This test case is expected to be suitable for smaller systems with up to 10 nodes.

\subsubsection{Carbon Fullerenes on a Lead Sheet}
This is a single-point energy calculation for a system consisting of two C60 fullerenes next to a Pb112 bulk sheet. The system consists of 232 atoms in a 14.2 x 14.2 x 40.0 Å unit cell. The calculation is performed using the residual minimisation method with the RMM-DIIS eigensolver and the Perdew-Burke-Ernzerhof exchange-correlation functional.

This test case is expected to be suitable for larger systems with up to 100 nodes. Smaller systems may be limited by the memory requirement, which can nevertheless be adjusted to some extent with the run parameters for Brillouin-zone sampling and grid spacing.


\section{GROMACS}
GROMACS is a versatile package to perform molecular dynamics, i.e. simulate the Newtonian equations of motion for systems with hundreds to millions of particles.
It is primarily designed for biochemical molecules like proteins, lipids and nucleic acids that have a lot of complicated bonded interactions, but since GROMACS is extremely fast at calculating the nonbonded interactions (that usually dominate simulations) many groups are also using it for research on non-biological systems, e.g. polymers.
GROMACS supports all the usual algorithms you expect from a modern molecular dynamics implementation, and some additional features:

GROMACS provides extremely high performance compared to all other programs. A lot of algorithmic optimisations have been introduced in the code; we have for instance extracted the calculation of the virial from the innermost loops over pairwise interactions, and we use our own software routines to calculate the inverse square root. In GROMACS 4.6, on almost all common computing platforms, the innermost loops are written in C using intrinsic functions that the compiler transforms to SIMD machine instructions, to utilise the available instruction-level parallelism. These kernels are available in either single and double precision, and support all different kinds of SIMD support found in x86-family processors available in January 2013.

\subsection{Code desctiption}
Parallelisation is achieved using combined OpenMP and MPI.
Offloading for accelerators is implemented through CUDA for GPGPUs and through OpenMP for MIC (Intel Xeon Phi).

GROMACS is written in C/C++ and freely available under the GPL license.

\subsection{Test cases desctiption}

\subsubsection{GluCL Ion Channel}
The ion channel system is the membrane protein GluCl, which is a pentameric chloride channel embedded in a lipid bilayer. The GluCl ion channel was embedded in a DOPC membrane and solvated in TIP3P water. This system contains 142k atoms, and is a quite challenging parallelisation case due to the small size. However, it is likely one of the most wanted target sizes for biomolecular simulations due to the importance of these proteins for pharmaceutical applications. It is particularly challenging due to a highly inhomogeneous and anisotropic environment in the membrane, which poses hard challenges for load balancing with domain decomposition.
This test case was used as the “Small” test case in previous 2IP and 3IP Prace phases. It is included in the package's version 5.0 benchmark cases. It is reported to scale efficiently up to 1000+ cores on x86 based systems??
Test case 2

\subsubsection{Lignocellulose}
A model of cellulose and lignocellulosic biomass in an aqueous solution [http://pubs.acs.org/doi/abs/10.1021/bm400442n]. This system of 3.3 million atoms is inhomogeneous. This system uses reaction-field electrostatics instead of PME and therefore scales well on x86. This test case was used as the “Large” test case in previous PRACE-2IP and -3IP projects. It is reported in previous PRACE projects to scale efficiently up to 10000+ x86 cores.

\section{NAMD}

NAMD is a widely used molecular dynamics application designed to simulate bio-molecular systems on a wide variety of compute platforms. NAMD is developed by the “Theoretical and Computational Biophysics Group” at the University of Illinois at Urbana Champaign. In the design of NAMD particular emphasis has been placed on scalability when utilising a large number of processors. The application can read a wide variety of different file formats, for example force fields, protein structure, which are commonly used in bio-molecular science. A NAMD license can be applied for on the developer’s website free of charge. Once the license has been obtained, binaries for a number of platforms and the source can be downloaded from the website. Deployment areas of NAMD include pharmaceutical research by academic and industrial users. NAMD is particularly suitable when the interaction between a number of proteins or between proteins and other chemical substances is of interest. Typical examples are vaccine research and transport processes through cell membrane proteins.

\subsection{Code desctiption}
NAMD is written in C++ and parallelised using Charm++ parallel objects, which are implemented on top of MPI, supporting both pure MPI and hybrid parallelisation.
See the Web site: http://www.ks.uiuc.edu/Research/namd/

Offloading for accelerators is implemented for both GPGPUs and MIC (Intel Xeon Phi).

\subsection{Test cases desctiption}
The datasets are based on the original "Satellite Tobacco Mosaic Virus (STMV)" dataset from the official NAMD site. The memory optimised build of the package and data sets are used in benchmarking. Data are converted to the appropriate binary format used by the memory optimised build.

\subsubsection{STMV.1M}
This is the original STMV dataset from the official NAMD site. The system contains roughly 1 million atoms. This data set scales efficiently up to 1000+ x86 Ivy Bridge cores.
Test case 2

\subsubsection{STMV.8M}
This is a 2x2x2 replication of the original STMV dataset from official NAMD site. The system contains roughly 8 million atoms. This data set scales efficiently up to 6000 x86 Ivy Bridge cores.

\section{PFARM}
PFARM is part of a suite of programs based on the ‘R-matrix’ ab-initio approach to the varitional solution of the many-electron Schrödinger equation for electron-atom and electron-ion scattering. The package has been used to calculate electron collision data for astrophysical applications (such as: the interstellar medium, planetary atmospheres) with, for example, various ions of Fe and Ni and neutral O, plus other applications such as data for plasma modelling and fusion reactor impurities. The code has recently been adapted to form a compatible interface with the UKRmol suite of codes for electron (positron) molecule collisions thus enabling large-scale parallel ‘outer-region’ calculations for molecular systems as well as atomic systems.

\subsection{Code desctiption}
In order to enable efficient computation, the external region calculation takes place in two distinct stages, named EXDIG and EXAS, with intermediate files linking the two. EXDIG is dominated by the assembly of sector Hamiltonian matrices and their subsequent eigensolutions. EXAS uses a combined functional/domain decomposition approach where good load-balancing is essential to maintain efficient parallel performance. Each of the main stages in the calculation is written in Fortran 2003 (or Fortran 2003-compliant Fortran 95), is parallelised using MPI and is designed to take advantage of highly optimised, numerical library routines. Hybrid MPI / OpenMP parallelisation has also been introduced into the code via shared memory enabled numerical library kernels.

Accelerator-based implementations have been implemented for both EXDIG and EXAS. EXAS uses offloading via MAGMA (or MKL) for sector Hamiltonian diagonalisations on Intel Xeon Phi and GPGPU accelerators. EXDIG uses combined MPI and OpenMP to distribute the scattering energy calculations on CPUs efficiently both across and within Intel Xeon Phi accelerators.

\subsection{Test cases desctiption}
External region R-matrix propagations take place over the outer partition of configuration space, including the region where long-range potentials remain important. The radius of this region is determined from the user input and the program decides upon the best strategy for dividing this space into multiple sub-regions (or sectors). Generally, a choice of larger sector lengths requires the application of larger numbers of basis functions (and therefore larger Hamiltonian matrices) in order to maintain accuracy across the sector and vice-versa. Memory limits on the target hardware may determine the final preferred configuration for each test case.

\subsubsection{FeIII}
This is an electron-ion scattering case with 1181 channels. Hamiltonian assembly in the coarse region applies 10 Legendre functions leading to Hamiltonian matrix diagonalisations of order 11810. In the fine region up to 30 Legendre functions may be applied leading to Hamiltonian matrices of order 35430. The number of sector calculations is likely to range from about 15 to over 30 depending on the user specifications. Several thousand scattering energies will be used in the calculation.

In the current model, parallelism in EXDIG is limited to the number of sector calculations, i.e around 30 accelerator nodes. Parallelism in EXAS is limited by the number of scattering energies, so we would expect this to reach into the hundreds of nodes.

\subsubsection{Methane}
The dataset is an electron-molecule calculation with 1361 channels. Hamiltonian dimensions are therefore estimated between 13610 and ~40000. The length of the external region required is relatively long, leading to more numerous sectors calculations (estimated to between 25 and 50). The calculation will require many thousands of scattering energies.

EXDIG scaling expected up to 50 accelerator nodes. EXAS scaling expected on hundreds to low thousands of nodes.

\section{QCD}

Matter consists of atoms, which in turn consist of nuclei and electrons. The nuclei consist of neutrons and protons, which comprise quarks bound together by gluons.

The theory of how quarks and gluons interact to form nucleons and other elementary particles is called Quantum Chromo Dynamics (QCD). For most problems of interest, it is not possible to solve QCD analytically, and instead numerical simulations must be performed. Such “Lattice QCD” calculations are very computationally intensive, and occupy a significant percentage of all HPC resources worldwide.

The MILC code is a freely-available suite for performing Lattice QCD simulations, developed over many years by a collaboration of researchers (physics.indiana.edu/~sg/milc.html).

The benchmark used here is derived from the MILC code (v6), and consists of a full conjugate gradient solution using Wilson fermions. The benchmark is consistent with “QCD kernel E” in the full UAEBS, and has been adapted so that it can efficiently use accelerators as well as traditional CPUs.

\subsection{Code desctiption}
The implementation for accelerators has been achieved using the “targetDP” programming model [http://ccpforge.cse.rl.ac.uk/svn/ludwig/trunk/targetDP/README], a lightweight abstraction layer designed to allow the same application source code to be able to target multiple architectures, e.g. NVidia GPUs and multicore/manycore CPUs, in a performance portable manner. The targetDP syntax maps, at compile time, to either NVidia CUDA (for execution on GPUs) or OpenMP+vectorisation (for implementation on multi/manycore CPUs including Intel Xeon Phi). The base language of the benchmark is C and MPI is used for node-level parallelism.

\subsection{Test cases desctiption}
Lattice QCD involves discretisation of space-time into a lattice of points, where the extent of the lattice in each of the 3 spatial and 1 temporal dimension can be chosen. This means that the benchmark is very flexible, where the size of the lattice can be varied with the size of the computing system in use (weak scaling) or can be fixed (strong scaling). For testing on a single node, then 64x64x32x8 is a reasonable size, since this fits on a single Intel Xeon Phi or a single GPU. For larger numbers of nodes, the lattice extents can be increased accordingly, keeping the geometric shape roughly similar.


\section{Quantum Espresso}

QUANTUM ESPRESSO is an integrated suite of computer codes for electronic-structure calculations and materials modelling, based on density-functional theory, plane waves, and pseudopotentials (norm-conserving, ultrasoft, and projector-augmented wave). QUANTUM ESPRESSO stands for opEn Source Package for Research in Electronic Structure, Simulation, and Optimisation. It is freely available to researchers around the world under the terms of the GNU General Public License. QUANTUM ESPRESSO builds upon newly restructured electronic-structure codes that have been developed and tested by some of the original authors of novel electronic-structure algorithms and applied in the last twenty years by some of the leading materials modelling groups worldwide. Innovation and efficiency are still its main focus, with special attention paid to massively parallel architectures, and a great effort being devoted to user friendliness. QUANTUM ESPRESSO is evolving towards a distribution of independent and inter-operable codes in the spirit of an open-source project, where researchers active in the field of electronic-structure calculations are encouraged to participate in the project by contributing their own codes or by implementing their own ideas into existing codes.
QUANTUM ESPRESSO is written mostly in Fortran90, and parallelised using MPI and OpenMP and is released under a GPL license.
Accelerator version

\subsection{Code desctiption}

During 2011 a GPU-enabled version of Quantum ESPRESSO was publicly released. The code is currently developed and maintained by Filippo Spiga at the High Performance Computing Service - University of Cambridge (United Kingdom) and Ivan Girotto at the International Centre for Theoretical Physics (Italy). The initial work has been supported by the EC-funded PRACE and a SFI (Science Foundation Ireland, grant 08/HEC/I1450). At the time of writing, the project is self-sustained thanks to the dedication of the people involved and thanks to NVidia support in providing hardware and expertise in GPU programming.

The current public version of QE-GPU is 14.10.0 as it is the last version maintained as plug-in working on all QE 5.x versions. QE-GPU utilised phiGEMM (external) for CPU+GPU GEMM computation, MAGMA (external) to accelerate eigen-solvers and explicit CUDA kernel to accelerate compute-intensive routines. FFT capabilities on GPU are available only for serial computation due to the hard challenges posed in managing accelerators in the parallel distributed 3D-FFT portion of the code where communication is the dominant element that limit excellent scalability beyond hundreds of MPI ranks.

A version for Intel Xeon PHI (MIC) accelerators is not currently available.

\subsection{Test cases desctiption}


\subsubsection{PW-IRMOF\_M11}
Full SCF calculation of a Zn-based isoreticular metal–organic framework (total 130 atoms) over 1 K point.  Benchmarks run in 2012 demonstrated speedups due to GPUs (NVidia K20s) with respect to non-accelerated nodes) in the range 1.37 – 1.87, according to node count (maximum number of accelerators=8). Runs with current hardware technology and an updated version of the code are expected to exhibit higher speedups (probably 2-3x) and scale up to a couple hundred nodes.


\subsubsection{PW-SiGe432}
This is a SCF calculation of a Silicon-Germanium crystal with 430 atoms. Being a fairly large system parallel scalability up to several hundred, perhaps a 1000 nodes is expected, with accelerated speed-ups likely to be of 2-3X.

\section{Synthetic benchmarks -- SHOC}
Code presentation
The Accelerator Benchmark Suite will also include a series of synthetic benchmarks. For this purpose, we choose the Scalable HeterOgeneous Computing (SHOC) benchmark suite, augmented with a series of benchmark examples developed internally. SHOC is a collection of benchmark programs testing the performance and stability of systems using computing devices with non-traditional architectures for general purpose computing. Its initial focus is on systems containing GPUs and multi-core processors, and on the OpenCL programming standard, but CUDA and OpenACC versions were added. Moreover, a subset of the benchmarks is optimised for the Intel Xeon Phi coprocessor. SHOC can be used on clusters as well as individual hosts.

The SHOC benchmark suite currently contains benchmark programs categorised based on complexity.  Some measure low-level 'feeds and speeds' behaviour (Level 0), some measure the performance of a higher-level operation such as a Fast Fourier Transform (FFT) (Level 1), and the others measure real application kernels (Level 2). The actual benchmarks for each level are listed below:

\subsection{Code desctiption}

All benchmarks are MPI-enabled. Some will report aggregate metrics over all MPI ranks, others will only perform work for specific rank.

Offloading for accelerators is implemented through CUDA and OpenCl for GPGPUs and through OpenMP for MIC (Intel Xeon Phi). For selected benchmarks OpenACC implementations are provided for GPGPUs. Multi-node parallelisation is achieved using MPI.

SHOC is written in C++ and is open-source and freely available: https://github.com/vetter/shoc .
Test cases definition
The benchmarks contained in SHOC currently feature 4 different sizes for increasingly large systems. The size convention is as follows:
1 - CPUs / debugging
2 - Mobile/integrated GPUs
3 - Discrete GPUs (e.g. GeForce or Radeon series)
4 - HPC-focused or large memory GPUs (e.g. Tesla or Firestream Series)

In order to go even larger scale we plan to add a 5th level for massive supercomputers.

\section{SPECFEM3D}
The software package SPECFEM3D simulates three-dimensional global and regional seismic wave propagation based upon the spectral-element method (SEM). All SPECFEM3D\_GLOBE software is written in Fortran90 with full portability in mind, and conforms strictly to the Fortran95 standard. It uses no obsolete or obsolescent features of Fortran77. The package uses parallel programming based upon the Message Passing Interface (MPI).

The SEM was originally developed in computational fluid dynamics and has been successfully adapted to address problems in seismic wave propagation. It is a continuous Galerkin technique, which can easily be made discontinuous; it is then close to a particular case of the discontinuous Galerkin technique, with optimised efficiency because of its tensorised basis functions. In particular, it can accurately handle very distorted mesh elements. It has very good accuracy and convergence properties. The spectral element approach admits spectral rates of convergence and allows exploiting hp-convergence schemes. It is also very well suited to parallel implementation on very large supercomputers as well as on clusters of GPU accelerating graphics cards. Tensor products inside each element can be optimised to reach very high efficiency, and mesh point and element numbering can be optimised to reduce processor cache misses and improve cache reuse. The SEM can also handle triangular (in 2D) or tetrahedral (3D) elements as well as mixed meshes, although with increased cost and reduced accuracy in these elements, as in the discontinuous Galerkin method.

In many geological models in the context of seismic wave propagation studies (except for instance for fault dynamic rupture studies, in which very high frequencies of supershear rupture need to be modelled near the fault, a continuous formulation is sufficient because material property contrasts are not drastic and thus conforming mesh doubling bricks can efficiently handle mesh size variations. This is particularly true at the scale of the full earth. Effects due to lateral variations in compressional-wave speed, shear-wave speed, density, a 3D crustal model, ellipticity, topography and bathyletry, the oceans, rotation, and self-gravitation are included. The package can accommodate full 21-parameter anisotropy as well as lateral variations in attenuation. Adjoint capabilities and finite-frequency kernel simulations are also included.

\subsection{Test cases definition}
The small test case runs with 16 MPI tasks, the large one runs with 7776 MPI tasks.


\part{Targeted architechture}
Description des machines K40/P100/ Xeons Phi

\part{Applications performance}
Présentation des résultats
topo comparatif des deux architechture

\part*{Conclusion}
open vers le merge des suite de bench prace
